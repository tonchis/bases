\section{Introducción}

\indent \indent El objetivo del presente Trabajo Práctico consistió en el análisis, modelado e implementación de un problema del mundo real, haciendo uso de las técnicas dadas en la materia y de un motor de bases de datos relacional.\\
\indent El problema en cuestión se trató de la realización de un sistema de voto electrónico.\\
\indent Dadas la descripción y los detalles del problema, se comenzó con un análisis preliminar, que consistió en discusiones de los grupo en base al enunciado.\\
\indent Finalizado el análisis, se continuó con el modelado del problema. Para ello, hicimos uso del Modelo Entidad-Relación (MER), realizando un Diagrama Entidad-Relación (DER) basándose en el entendimiento del problema extraído del análisis. En base a esto se comenzó, además, a definir distintas restricciones a nuestro modelo que el poder expresivo del DER no nos permitía explicitar.\\
\indent Una vez finalizado preliminarmente el DER, se hicieron uso de las técnicas dadas por la cátedra para contruir el Modelo Relacional derivado del MER confeccionado.\\
\indent Concluida la sección de modelado, se procedió a la etapa implementativa del trabajo. Decidimos utilizar el motor de bases de datos PostGreSQL para dicho fin.\\
\indent Durante esta etapa se contruyó el diseño físico de la solución propuesta, además de la implementación de las funcionalidad pedidas en el enunciado del Trabajo Práctico, haciendo uso de Stored Procedures, y de la restricciones anteriormente discernidas, en este caso usando Triggers.\\
\indent Todo este proceso, que a priori parece secuencial, tuvo un carácter más bien iterativo, puesto que orgánicamente fueron surgiendo diversas realizaciones sobre el problema y sobre la solución propuesta que impactaron en etapas que ya se habían realizado.\\
