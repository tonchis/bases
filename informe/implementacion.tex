\section{Implementación}


Para poder verificar que la cantidad de votos no se mayor a la cantidad de gente que votó se creo un trigger que verifica que la suma de  cantidad de votos de todos los  candidatos en una MesaCandidato no supere a la suma de Ciudadanos que votaron en la Mesa. Como puede haber ciudadanos que no voten o que voten en blanco no se verifica la igualdad. 
    El trigger obliga  a que primero se tenga que actualizar la hora de voto en VotaEn para poder agregar el voto a cantidad. 

\subsection{Triggers}

\subsubsection{}

\begin{verbatim}
CREATE OR REPLACE FUNCTION OriundaDeCiudad() RETURNS TRIGGER AS $$
    BEGIN
    IF (SELECT count(*) 
        FROM Territorio 
        WHERE idTerritorio=NEW.idTerritorio AND idTipoTerritorio=3) = 0 
        THEN
            RAISE EXCEPTION 'Solo puede ser oriundo de una ciudad';              
        ELSE
            RETURN NEW;
    END IF;
    RETURN NULL;
    END;
$$ LANGUAGE plpgsql;


CREATE TRIGGER CheckInsertPersona
    BEFORE INSERT ON Persona
    FOR EACH ROW EXECUTE PROCEDURE OriundaDeCiudad();

\end{verbatim}

\subsubsection{}
\begin{verbatim}

CREATE OR REPLACE FUNCTION checkVotosCandidato() RETURNS TRIGGER AS $$
    BEGIN
    IF (SELECT count(*) 
        FROM MesaElectoral M 
        INNER JOIN Eleccion E ON M.idEleccion = E.idEleccion
        INNER JOIN SePostula S on E.idEleccion =  S.idEleccion
        WHERE S.idCandidato = NEW.idCandidato 
        AND M.idMesaElectoral = NEW.idMesaElectoral ) = 0
        THEN
            RAISE EXCEPTION 
            'El candidato se ha postulado para la eleccion de la 
            Mesa';              
        ELSE
            RETURN NEW;
    END IF;
    RETURN NULL;
    END;
$$ LANGUAGE plpgsql;


CREATE TRIGGER CheckInsertVotosCandidato
    BEFORE INSERT ON VotosCandidato
    FOR EACH ROW EXECUTE PROCEDURE checkVotosCandidato();


CREATE TRIGGER CheckUpdateVotosCandidato
    BEFORE UPDATE ON VotosCandidato
    FOR EACH ROW EXECUTE PROCEDURE checkVotosCandidato();


\end{verbatim}
\subsection{Stored Procedures para funcionalidades pedidas}

\subsubsection{Obtener los ganadores del las elecciones transcurridas en el último año}

\indent Se consideró que las elecciones del último año son aquellas que se realizaron a lo sumo hasta un año antes de la fecha de la consulta.\\

\begin{verbatim}
CREATE OR REPLACE FUNCTION GetGanadoresUltimoAnio()
RETURNS TABLE(idEleccion INTEGER, Nombre CHARACTER VARYING(510),
              TipoDocumento CHARACTER VARYING(255), NumeroDocumento INT) AS

$BODY$

SELECT e.idEleccion, p.Nombre||''||p.Apellido, d.Tipo, d.Numero
FROM Eleccion e
INNER JOIN MesaElectoral me ON e.idEleccion = me.idEleccion
INNER JOIN VotosCandidato vc ON me.idMesaElectoral = vc.idMesaElectoral
INNER JOIN Persona p ON p.idPersona = vc.idCandidato
INNER JOIN Documento d ON d.idPersona= p.idPersona
WHERE (e.Fecha BETWEEN (now()-INTERVAL '365 days') AND now())
GROUP BY e.idEleccion, p.Nombre,p.Apellido,d.Tipo, d.Numero
HAVING NOT EXISTS
    (SELECT sum(vc2.Cantidad)
    FROM Eleccion e2
    INNER JOIN MesaElectoral me2 ON me2.idEleccion = e2.idEleccion
    INNER JOIN MesaCandidato mc2 ON me2.idMesaElectoral = mc2.idMesaElectoral
    INNER JOIN VotosCandidato vc2 ON vc2.idMesaElectoral = mc2.idMesaElectoral
    WHERE e2.idEleccion = e.idEleccion
    GROUP BY vc2.idCandidato
    HAVING SUM(vc2.Cantidad) > SUM(vc.Cantidad))

$BODY$

LANGUAGE sql;

\end{verbatim}

\subsubsection{Consultar las cinco personas que más tarde fueron a votar antes de terminar la votación por cada centro electoral en una elección}

\begin{verbatim}
CREATE OR REPLACE FUNCTION GetVotantesQueMasTardeVotaronPorCentroElectoral(eleccionID INTEGER)
RETURNS TABLE(idCentroVotacion INTEGER,
              DireccionCentroElectoral CHARACTER VARYING(255),
              Nombre CHARACTER VARYING(510), TipoDocumento CHARACTER
              VARYING(255), NumeroDocumento INT) AS

$BODY$

SELECT cv.idCentroVotacion, cv.Direccion, p.Nombre||' '||p.Apellido, d.Tipo, d.Numero 
FROM Eleccion e
INNER JOIN  MesaElectoral me ON me.idEleccion = e.idEleccion
INNER JOIN CentroVotacion cv ON cv.idCentroVotacion = me.idCentroVotacion
INNER JOIN VotaEn ve ON me.idMesaElectoral = ve.idMesaElectoral
INNER JOIN Ciudadano c ON ve.idCiudadano = c.idPersona
INNER JOIN Persona p ON c.idPersona =  p.idPersona
INNER JOIN Documento d ON d.idPersona = p.idPersona
WHERE e.idEleccion = eleccionID
AND c.idPersona IN (
    SELECT ve2.idCiudadano
    FROM CentroVotacion cv2
    INNER JOIN MesaElectoral me ON cv2.idCentroVotacion = me.idCentroVotacion
    INNER JOIN VotaEn ve2 ON me.idMesaElectoral = ve2.idMesaElectoral
    WHERE cv2.idCentroVotacion = cv.idCentroVotacion
    ORDER BY ve2.fecha, ve2.hora DESC
    LIMIT 5
  )
ORDER BY cv.idCentroVotacion, ve.fecha, ve.hora

$BODY$

LANGUAGE sql;

\end{verbatim}

\subsubsection{Consultar quiénes fueron los partidos políticos que obtuvieron más del 20\% en las últimas cinco elecciones provinciales a gobernador}

\begin{verbatim}
CREATE OR REPLACE FUNCTION GetPartidosQuePasaronLos20EnLasUltimasCincoParaGobernador()
RETURNS TABLE(PartidoPolitico CHARACTER VARYING (255)) AS
$BODY$
SELECT Nombre FROM (
    SELECT e.idEleccion , pp.Nombre
    FROM Eleccion e
    INNER JOIN MesaElectoral me ON e.idEleccion = me.idEleccion
    INNER JOIN MesaCandidato mc ON me.idMesaElectoral = mc.idMesaElectoral
    INNER JOIN VotosCandidato vc ON vc.idMesaElectoral = mc.idMesaElectoral
    INNER JOIN Candidato c ON c.idPersona = vc.idCandidato 
    INNER JOIN Persona p ON p.idPersona = c.idPersona
    INNER JOIN SePostula sp ON sp.idCandidato = c.idPersona and sp.idEleccion = e.idEleccion
    INNER JOIN PartidoPolitico pp ON pp.idPartidoPolitico = sp.idPartidoPolitico
    WHERE e.idEleccion IN (
        SELECT idEleccion
        FROM Eleccion 
        INNER JOIN TipoEleccion ON Eleccion.idTipoEleccion = TipoEleccion.idTipoEleccion
        WHERE TipoEleccion.idTipoEleccion = 2
        ORDER BY Eleccion.Fecha DESC
        LIMIT 5)
    GROUP BY e.idEleccion,pp.idPartidoPolitico, pp.Nombre
    HAVING ((SUM(vc.Cantidad)) *100 /
            (SELECT SUM(vc2.Cantidad) 
             FROM Eleccion e2 
             INNER JOIN MesaElectoral me2 ON e2.idEleccion = me2.idEleccion
             INNER JOIN VotosCandidato vc2 ON me2.idMesaElectoral = vc2.idMesaElectoral
             WHERE e2.idEleccion = e.idEleccion) > 20)
    ) AS PartidosConMasDe20PorEleccion
    GROUP BY Nombre
    HAVING COUNT(Nombre) =  CASE WHEN (SELECT COUNT(*) FROM Eleccion WHERE idTipoEleccion = 2) > 5 
                THEN 5 
                ELSE (SELECT COUNT(*) FROM Eleccion WHERE idTipoEleccion = 2)
                END  
$BODY$

LANGUAGE sql;

\end{verbatim}