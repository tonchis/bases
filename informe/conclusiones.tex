\section{Conclusiones}

\indent Durante la confección del presente Trabajo Práctico hemos pasado por distintas etapas para proponer una solución a partir de un problema real.\\
\indent Se comenzó con el modelado del problema, abstrayendo aquellos conceptos y relaciones que nos parecieron de importancia para encarar la solución. Es discutible si esta primera etapa finaliza en algún momento, puesto que el descubrimiento de nuevos supuestos, las decisiones que se van tomando durante el proceso y la validación de lo realizado suponen cambios constantemente, incluso al momento de realizar la implementación. En numerosas oportunidades nos hemos visto obligados a realizar cambios en el modelo, incluso en los estadíos más tardíos del proceso. Esto trae como consecuencia que el proceso de realización del trabajo haya tenido una forma cíclica, dónde siempre se volvió a revisar los pasos anteriores al momento de analizar alguna potencial modificación. \\
\indent Se extrae de esto la importacia del entendimiento del problema y de su contexto, asi como de las funcionalidades requeridas, al momento de construir un modelo que abstraiga de manera adecuada al sujeto de análisis. Mientras más tempranas sean estas realizaciones, menor será el costo de realizar los cambios correspondientes, y se podrá continuar sobre una base más firme el proceso de implementación de la base de datos.\\
\indent Se pudo observar además, la fuerte relación existente entre el Modelo Entidad-Relación, el Modelo Relacional y la implementación de la base de datos en el motor elegido. Sin embargo, el acotado poder expresivo de los modelos nos obligó a pensar en restricciones adicionales, explicitadas en lenguaje natural a la hora del modelado e implementadas mayoritariamente a través de Triggers y Funciones, haciendo uso de las herramientas del motor de bases de datos elegido, PostGreSQL.\\
\indent Podríamos preguntarnos, una vez finalizado el trabajo y al observar la gran cantidad de restricciones que debimos implementar, si un enfoque ligeramente distinto a la hora de confeccionar el Modelo Entidad-Relación nos podría haber simplificado muchas de las restricciones impuestas de esta manera. Creemos, sin embargo, que no existe modelado libre de problemas: es posible que otra propuesta de modelo solucione nuestras complejidades, pero traiga consigo muchas otras que nosotros no tuvimos que contemplar.\\   
